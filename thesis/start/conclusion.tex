%conclusion

I believe that GenSequence's best application is for testing database-driven applications. These applications literally control the world. Consider financial applications alone: the record of every transaction, every payment, every credit report, and every bill can dictate a person’s entire life. The infrastructure surrounding that data must not expunge or fabricate any of it, and must maintain its integrity. Software that accesses, controls, and manipulates this huge amount of data carries substantial responsibility. Moreover, software that can read, interpret, and identify trends in a huge wealth of global data has amazing power in informing us of what happens in the world and how we can make better decisions. It is therefore of utmost importance to design that software well enough to trust its results.

GenSequence has addressed those applications. It functions simply but provides automation during the testing stage and generates reliable data that is what is says it is. If the context of the program warrants consideration of statistically unlikely but possible scenarios, GenSequence can make that happen. Moreover, the construction of data points It primarily states its power by its ease of use and nearly end-to-end automation.

Given that the construction of GenSequence was a Minimum Viable Product, there is considerable room for improvement.

\textbf{Current Iteration}
\begin{enumerate}
\item Identify an open-source project in need of a test suite, and create a hypothetical test suite for it using GenSequence.
\end{enumerate}

\textbf{Top of the Backlog}
\begin{enumerate}
\item Implement the entire preprocessor in Ply, develop an abstraction for a model of the Parms, and develop an interpreter that could execute those models.
\item Combine the specifications that enter GenPairs and the preprocessor into one. This would certainly reduce the work for the tester.
\end{enumerate}
Merge-sort .cp and .prm files, since it makes logical sense that they be in the same file
Double Cardioid distribution using hill climbing


Icebox
Actually write documentation
deeper understanding on how random works, and why
ML application, see what/how GenSequence could be used
