%methods

One key part of this project is deciding if it accomplished its goal and answered its research questions. 
\begin{enumerate}
\item Is using random statistical distributions a valid approach to writing test data?
\item Do symbolic test vectors provide enough information about the input that it informs expectations of the output?
\end{enumerate}

Testers might use this tool at any stage during the software development cycle. Perhaps their application is highly advanced, has been neatly designed, and has fixed bugs found through other means of testing or from end user reports. Therefore, that piece of software simply may be bug--free. That is unlikely, but it is possible. So measuring the usefulness of GenSequence on fault finding ability alone would depend on the state of the software it is testing, giving the indication that GenSequence is inadequate when really the software is just really well done.

In response to Question 2, the intention of this project was to streamline the testing process, and provided automated methods of data creation. The usefulness of this program is measured by the ease of test suite creation. I will consider these metrics: the length of the script generating data, how much code a tester would have to write, and how quickly the data is generated (more data takes more time). Ultimately it is a tester's decision as to how useful this project is, but these goals become a self-fulfilling prophecy. More features added to the project only increase the ease of use, since a machine does more of the heavy lifting than a human brain and body otherwise would have. Nevertheless, there is further discussion later.

Finally, it is important to determine how valid the random statistical distribution approach is. I will use a few case studies to determine if knowing how a parameterss data was generated helps clarify the expected result. First I will test my testing tool against a planetary orbits simulation. Then I will observe my tool's data representation in an earthquake analysis program. 