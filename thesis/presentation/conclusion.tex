\section{Concluding Thoughts}
\frame{
\frametitle{GenSequence in all its Power}
\begin{itemize}
\item It is what it says it is
\item Nearly end-to-end automation
\item Knowing about input informs the expected output?
\end{itemize}
}


\frame{
\frametitle{Future Work}
\begin{itemize}
\item Test GenSequence against an open-source project
\item Combine user-written spec files
\item Machine Learning Models
\item Database-Driven Applications
\end{itemize}
}

\frame{
\frametitle{Citations}
\begin{itemize}
\item \tiny Robins, Annabel. Software Testing and QA: Theory of Practice. Univeristy of Waterloo: Naik and Tripathy, 2009. http://slideplayer.com/slide/5946428/20/images/10/Pairwise+Testing+Handouts.+The+total+number+of+all-combination+test+cases+is+2+\%C3\%97+2+\%C3\%97+2+=+8..jpg
\item N. G. Leveson and C. S. Turner. 1993. An Investigation of the Therac-25 Accidents. Computer 26, 7 (July 1993), 18-41. DOI: https://doi.org/10.1109/MC.1993.274940
\item Klein, Dave. ``The Scientific Method.'' Brooklyn College BIOL 1010 : Lab Notes. Brooklyn College, 2016.
\end{itemize}
}