%%% This is an example file for the Auburn University style options
%%%       aums.sty (Masters Thesis)
%%%       auphd.sty (Ph.D. Dissertation)
%%%       auhonors.sty (Honors Scholar)

%%%To use it, please edit the necessary options, title, author, date, year, keywords, advisor, professor, etc. 

\documentclass[12pt]{report}
%\usepackage{aums}       % For Master's papers
%\usepackage{auphd}     % For Ph.D.
\usepackage{auhonors}  % For honors college
\usepackage{ulem}       % underlining on style-page; see \normalem below
\usepackage{url}
\usepackage{tikz}
\usepackage{pgf}
\usepackage{abstract}
%%%%%Format rules: Normal margins are 1 in. If you need to print with 1.5in margins, uncomment the line below
\oddsidemargin0.5in \textwidth6in

%% If you do not need a List of Abbreviations, then comment out the lines below and the \printnomenclature line.
%%for List of Abbreviations information:  (see http://www.mackichan.com/TECHTALK/509.htm  )
\usepackage[intoc]{nomencl}
\renewcommand{\nomname}{List of Abbreviations}   	       
\makenomenclature 
%% don't forget to run:   makeindex ausample.nlo -s nomencl.ist -o ausample.nls
%% Also, if 




% May want theorems numbered by chapter
\newtheorem{theorem}{Theorem}[chapter]


\begin{document}



\begin{titlepage}
    \begin{center}
        \vspace*{1cm}
        
        \Large
        \textbf{USING STATISTICAL DISTRIBUTIONS FOR GENERATING RANDOM TEST DATA}
        
        \vspace{3.5cm}
        
        \large
        by
        \linebreak
        JAMIE ZIMMERMAN
        
        \vspace{3.5cm}
        
        \vfill

        \normalsize
        A THESIS
        \vspace{1.5cm}
        \begin{singlespace}
        Presented to the Department of Computer and Information Science \\
        and the Robert D. Clark Honors College \\
        in partial fulfillment of the requirements for the degree of \\
        Bachelor of Science
        \end{singlespace}
        
        \vspace{0.8cm}
        June 2018
        
    \end{center}
\end{titlepage}



\begin{romanpages}      % roman-numbered pages 


\chapter*{\textbf{An Abstract of the Thesis of}}
%abstract
\thispagestyle{plain}
\begin{center}
%\large\textbf{An Abstract of the Thesis of}

%\vspace{.75cm}

\normalsize Jamie Zimmerman for the degree of Bachelor of Arts \\
in the Department of Computer and Information Science to be taken June 2018.

\vspace{2cm}


Title: Using Statistical Distributions to Generate Random Test Data

\vspace{1.5cm}

Approved: \hrulefill

Dr. Michal Young

\end{center}
Here is my abstract.
 \vfill



\begin{acknowledgments}
My acknowledgements to Michal and friends.
\end{acknowledgments}

\tableofcontents
\vfill

%list of materials
%materials
\thispagestyle{plain}

\newpage
\begin{center}

\textbf{List of Accompanying Materials}

\vspace{.5cm}

\begin{enumerate}
\item GenSequence: https://github.com/TestCreator/GenSequence \\
\item GenPairs: https://github.com/TestCreator/GenPairs
\end{enumerate}
\end{center}



\listoffigures
\listoftables
\printnomenclature[0.5in] %used for the List of Abbreviations
\end{romanpages}        % All done with roman-numbered pages%

\normalem       % Make italics the default for \em%

\chapter{This is the Title of the\\ First Chapter}  % Use \\ for long titles %
This is a sample document for the Auburn \nomenclature{Auburn}{Auburn University} \LaTeX{} style-files known
as {\tt aums} (for Master's papers) and {\tt auphd} (for Ph.D.'s).
The appendix contains some of the history of this project, including
contact information for the authors. 
Site administrators should upgrade to \LaTeX2e; however, the style files
should work with the older \LaTeX.  \vfill The style files should be available
on mallard.  The current release is available by anonymous ftp to
ftp.dms.auburn.edu in the directory aums (on-campus computers may
also retrieve these from \url{http://www.dms.auburn.edu/manuals}).
Most users will need either Lamport's book \cite{lamport} or Hahn's book
\cite{hahn}.%
If you do not need the List of Abbreviations\nomenclature{LoA}{List of Abbreviations}, comment the nomencl package and associated nomenclature commands. 
\begin{theorem} This is an example theorem.
\end{theorem}%
\section{This is an example of a section heading}%
This is some text which follows the section heading. You can find the data in Table \ref{tab:results}.
\begin{figure}
  \begin{center}
  \setlength{\unitlength}{.7in}
    \begin{picture}(4.2,1)
      \put(0.2,.5){\circle{0.1}}
      \put(0.9,.5){\circle{0.2}}
      \put(1.6,.5){\circle{0.3}}
      \put(2.3,.5){\circle{0.4}}
      \put(3.1,.5){\circle{0.5}}
      \put(3.8,.5){\circle{0.6}}
    \end{picture}
  \end{center}
 \caption{Hollow circles 1}\label{HollowCircles}
\end{figure}%
\begin{figure}
\centering
  \begin{tikzpicture}
\filldraw [double distance=0.20mm,very thick, fill=white, draw=black] (0cm,0cm) rectangle (5cm, 2.5cm);
   \fill (2.5in,0in) circle (3pt);
\end{tikzpicture}
\caption{Some TikZ picture.}
\end{figure}%
\begin{table}[htb]
\begin{center}
\begin{tabular}{|c | c | c | c|}
\hline
\multicolumn{1}{|c|}{~} & \multicolumn{3}{c|}{Multicolumn Heading 1}\\
Heading 1 & \multicolumn{1}{|c}{Heading 2} & \multicolumn{1}{c}{Heading 3} & \multicolumn{1}{c|}{Heading 4} \\
\hline
1 & 19, 20 (19.5)& NA & NA \\
\hline
3 & $\infty$* ($\infty$)& 18, 15 (16.5)& 9, 9 (9)\\
\hline
5 & 23, 18 (20.5) 
& 16 (16)
& 7, 7, 8 (7.33)\\
\hline
\multicolumn{4}{|c|}{*Some random comment for the whole table.}\\
\hline
\end{tabular}
\end{center}
\caption{Some Table of data}
\label{tab:results}
\end{table}%

\subsection{This is a subsection heading}%
Text after the subsection. And we have a figure, Figure \ref{HollowCircles}.
\chapter{New Chapter}
\begin{theorem}
Another theorem.
\end{theorem}%
\begin{figure}
  \begin{center}
  \setlength{\unitlength}{.7in}
    \begin{picture}(4.2,1)
      \put(0.2,.5){\circle{0.1}}
      \put(0.9,.5){\circle{0.2}}
      \put(1.6,.5){\circle{0.3}}
      \put(2.3,.5){\circle{0.4}}
      \put(3.1,.5){\circle{0.5}}
      \put(3.8,.5){\circle{0.6}}
    \end{picture}
  \end{center}
 \caption{Hollow circles}\label{HollowCircles}
\end{figure}
%%%%%%%%Two options for having a bibliography. If you use a separate file or multiple files:
%\bibliography{./robotics,./imageprocessing,./thesis}
%%%%% where the files are robotics.bib, imageprocessing.bib and/or thesis.bib. %
%Or you can include the bibliography entries directly:%
\begin{thebibliography}{99}
\newcommand{\AmS}{$${\protect\the\textfont2 A}\kern-.1667em\lower
         .5ex\hbox{\protect\the\textfont2 M}\kern
         -.125em{\protect\the\textfont2 S}}%
\bibitem{hahn} Jane Hahn, ``\LaTeX{} For Everyone,'' Personal TeX Inc., 
  12 Madrona Street, Mill Valley, California.%
\bibitem{goossens} Frank Mittelbach and Michel Goossens (with
Johannes Braams, David Carlisle, and Chris Rowley),
  ``The \LaTeX{} Companion,'' second edition, Addison-Wesley, 2004.%
\bibitem{GRM:LaTeXGraphicsCompanion} 
Michal Goossens, Sabastian Rahtz, and Frank Mittelbach, ``The \LaTeX{} 
Graphics Companion,'' Addison-Wesley, 1997.%
\bibitem{Gra:MathIntoLaTeX} George Gr\"atzer, ``Math into \LaTeX: An 
introduction to \LaTeX{} and \mbox{\AmS-\LaTeX},'' Birkh\"auser, 1996.%
\bibitem{hoenig} Alan Hoenig, ``\TeX{} Unbound: \LaTeX{} and \TeX{} strategies
for Fonts, Graphics, and More,'' Oxford University Press, 1997. Includes
practical advice and numerous exammples for a wide range of topics,
includeing virtual fonts, graphics, and resources for the internet and
multimedia.%
\bibitem{KH} Helmut Kopka and Patrick W. Daly, ``A Guide to \LaTeXe:
  Document Preparation for Beginners and Advanced Users,'' 2nd ed.,
  Addison-Wesley, 1995.%
\bibitem{lamport} Leslie Lamport, ``\LaTeX: A Document Preparation
  System,'' 2nd ed., Addison-Wesley, 1994.%
\bibitem{walsh} Norman Walsh, ``Making \TeX{} Work,'' O'Reilly and
  Associates, 1994.%
\end{thebibliography}%


\end{document}

